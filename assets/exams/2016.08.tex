\section{Agosto 2016}

\subsection{Testo}

Si consideri una possibile implementazione degli insiemi di interi in standard ML, in cui un insieme di interi  rappresentato da una funzione da \sml{int} a \sml{bool}:

\begin{lstlisting}[style = SML, caption = {Definizione del tipo di dato \sml{insiemidiinteri}}]
type insiemediinteri = int -> bool;
\end{lstlisting}

La funzione applicata ad un numero intero ritorna true se il numero appartiene all'insieme, false altrimenti. L'insieme vuoto è quindi rappresentato da una funzione che ritorna sempre false:

\begin{lstlisting}[style = SML, caption = {Definizione della funzione \sml{vuoto}}]
val vuoto:insiemediinteri = fn n => false;

val vuoto = fn: insiemediinteri
\end{lstlisting}

ed un intero può essere aggiunto ad un insieme tramite la funzione aggiungi:

\begin{lstlisting}[style = SML, caption = {Definizione della funzione \sml{aggiungi}}]
val aggiungi = fn f:insiemediinteri => fn x:int =>
	(fn n:int => if (n = x) then
					true
				else
					false
	):insiemediinteri;

val aggiungi = fn: insiemediinteri -> int -> insiemediinteri
\end{lstlisting}

\'E possibile verificare se un intero è contenuto in un insieme tramite la funzione contiene:

\begin{lstlisting}[style = SML, caption = {Definizione della funzione \sml{contiene}}]
val contiene = fn f:insiemediinteri => fn n:int => f n;
\end{lstlisting}

Si implementi la funzione \sml{intersezione}, avente tipo \sml{insiemediinteri -> insiemediinteri -> insiemediinteri}, che dati due insiemi di interi ne calcola l'intersezione.

\textbf{IMPORTANTE}: notare il tipo della funzione! Come si può intuire da tale tipo, usa la \emph{tecnica del currying} per gestire i suoi due argomenti.

\subsection{Soluzione}

\begin{lstlisting}[style = SML, caption = {Definizione della funzione \sml{intersezione}}]
val intersezione = fn i1:insiemediinteri => fn i2:insiemediinteri =>
	(fn n =>
		((contiene i1 n) andalso (contiene i2 n))
	):insiemediinteri;

val intersezione = fn: insiemediinteri -> insiemediinteri -> insiemediinteri
\end{lstlisting}

\subsection{Commento della soluzione}

Si può arrivare alla soluzione per passi analizzando

\begin{lstlisting}[style = SML]
((contiene i1 n) andalso (contiene i2 n))
\end{lstlisting}

L'operatore \sml{andalso} implementa l'operazione logica \emph{and}, mentre l'operatore \sml{orelse} implementa l'operazione logica \emph{or}.
