\section*{Introduzione}

Lo scopo principale di questi appunti è quello di esaminare più da vicino gli esami di programmazione funzionale tenuti all'Università degli Studi di Trento. Queste note non sono complete, e la loro lettura non permette, da sola, di superare l’esame. La versione più recente di queste note si trova all'indirizzo:

\begin{center}
	\url{https://github.com/emanuelenardi/latex-sml}
\end{center}

\subsection*{Materiale}

Nel commento della soluzione è possibile trovare l'esempio di esame pronto da compilare online.

Puoi trovare una veloce introduzione ad SML su Learn X in Y minutes %
\href{https://learnxinyminutes.com/docs/standard-ml/}{\ExternalLink}.

Ho prodotto una playlist di youtube che tratta gli argomenti del corso %
\href{bit.ly/sml-youtube-playlist}{\ExternalLink}.

Se trovi qualche video eplicativo e pensi che possa ritornare utile ai tuoi compagni di corso, tramite questo link, puoi aggingerli direttamente dal link sopra.

Per tutto il resto consulta la cartella Google Drive del corso di Informatica %
\href{https://bit.ly/drive-folder}{\ExternalLink}.

\subsection*{Segnalazione di errori}

Se hai trovato un errore ti prego di inviarmi un'e-mail \href{mailto:emanuele.nardi@studenti.unitn.it}{\ExternalLink} allegando un esempio che possa riprodurre l'errore.

\subsection*{Ringraziamenti}

Vorrei ringraziare particolamente a \href{https://github.com/mfranzil}{Matteo Franzil} e a \href{https://github.com/matteocontrini}{Matteo Contrini} per aver contribuito a migliorare questa dispensa.
