\section{Luglio 2016}

\subsection{Turno 1}

\subsubsection{Testo}

Si consideri il tipo di dato

\begin{lstlisting}[style = SML, caption = {Definizione del tipo di dato \sml{espressione Lambda}}]
datatype lambda_expr = Var of string
					 | Lambda of string * lambda_expr
					 | Apply of lambda_expr * lambda_expr;
\end{lstlisting}

che rappresenta un'espressione del Lambda-calcolo.

\medskip
Il costruttore \sml{Var} crea un'espressione costituita da un'unica funzione/variabile (il cui nome è un valore di tipo \sml{string});
il costruttore \sml{Lambda} crea una Lambda-espressione a partire da un'altra espressione, legandone una variabile (indicata da un valore di tipo string);
il costruttore \sml{Apply} crea un'espressione data dall'applicazione di un'espressione ad un'altra.

\medskip
Si scriva una funzione \sml{is\_free} (avente tipo \sml{string -> lambda\_expr -> bool}) che riceve come argomenti una stringa (che rappresenta il nome di una variabile / funzione) ed una Lambda-espressione, %
ritornando \sml{true} se \emph{la variabile indicata appare come libera nell'espressione}, %
\sml{false} altrimenti (quindi, la funzione ritorna false se la variabile è legata o se non appare nell'epressione).

\medskip
Come esempio, l'invocazione

\begin{lstlisting}[style = SML]
is_free "a" (Var "a")
\end{lstlisting}

deve avere risultato true, l'invocazione

\begin{lstlisting}[style = SML]
is_free "b" (Var "a")
\end{lstlisting}

deve avere risultato false, l'invocazione

\begin{lstlisting}[style = SML]
is_free "a" (Lambda ("a", Apply((Var "a"), Var "b")))
\end{lstlisting}

deve avere risultato false, l'invocazione

\begin{lstlisting}[style = SML]
is_free "b" (Lambda ("a", Apply((Var "a"), Var "b")))
\end{lstlisting}

deve avere risultato true e così via.

\medskip
\textbf{IMPORTANTE}: notare il tipo della funzione! La funzione usa la \emph{tecnica del currying} per gestire i due argomenti.

\subsubsection{Soluzione}

\begin{lstlisting}[style = SML, nolol = true, caption = {Definizione del tipo di dato \sml{espressione Lambda}}]
datatype lambda_expr = Var of string
					 | Lambda of string * lambda_expr
					 | Apply of lambda_expr * lambda_expr;
\end{lstlisting}

\begin{lstlisting}[style = SML, caption = {Definizione della funzione \sml{is\_free}}]
val rec is_free = fn s => fn Var v => if (s = v) then
									      true
									  else
									      false
				| Lambda (v, e) => if (s = v) then
									   false
								   else
									   is_free s e
				| Apply (e1, e2) => (is_free s e1) orelse (is_free s e2);

val is_free = fn: string -> lambda_expr -> bool
\end{lstlisting}

\subsection{Commento della soluzione}

\omissis

% ---------------------------------------------------------------------------- %

\subsection{Turno 2}

\subsubsection{Testo}

Basandosi sul tipo di dato \sml{espressione} e la funzione \sml{eval} definiti come segue:

\begin{lstlisting}[style = SML, caption = {Definizione della funzione \sml{semplifica}}]
local
	val rec eval = fn costante		n        => n
					| somma			(a1, a2) => (eval a1) + (eval a2)
					| sottrazione	(a1, a2) => (eval a1) - (eval a2)
					| prodotto		(a1, a2) => (eval a1) * (eval a2)
					| divisione		(a1, a2) => (eval a1) div (eval a2);
in
	val semplifica = fn costante	n        => costante(n)
					  | somma		(a1, a2) => costante((eval a1) + (eval a2))
					  | sottrazione	(a1, a2) => costante((eval a1) - (eval a2))
					  | prodotto	(a1, a2) => costante((eval a1) * (eval a2))
					  | divisione	(a1, a2) => costante((eval a1) div (eval a2))
end;

(tipo funzione)
\end{lstlisting}

il tipo espressione può essere esteso come segue per supporare il concetto di variabile:

\begin{lstlisting}[style = SML, caption = {Definizione del tipo di dato \sml{espressione}}]
datatype espressione = costante    of int
					 | variabile   of string
					 | somma       of espressione * espressione
					 | sottrazione of espressione * espressione
					 | prodotto    of espressione * espressione
					 | divisione   of espressione * espressione
					 | var         of string      * espressione * espressione;
\end{lstlisting}

Si riscriva la funzione \sml{eval} per supportare i due nuovi costruttori \sml{variabile} e \sml{var. variabile x}, con \sml{x} di tipo stringa, é valutata al valore della variabile di nome \sml{x} (per fare questo, \sml{eval} deve cercare nell'ambiente un legame fra tale nome ed un valore). %
\sml{var (x, e1, e2)} é valutata al valore di \sml{e2} dopo aver assegnato ad \sml{x} il valore di \sml{e1}.

\medskip
Per poter valutare correttamente \sml{variabile} e \sml{var}, \sml{eval} deve quindi ricevere come argomento l'ambiente in cui valutare le variabili. %
Tale ambiente può essere rappresentato come una lista di coppie (stringa, intero) ed avra' quindi tipo \sml{(string * int) list}.

\medskip
La funzione \sml{eval} deve quindi avere tipo \sml{(string * int) list -> espressione -> int}.

\subsubsection{Soluzione}

Questa é una possibile soluzione. %
Si noti che in questa soluzione la funzione \sml{cerca} viene definita come visibile a tutti, mentre sarebbe più opportuno renderla locale a \sml{eval} usando un costrutto \sml{let} o \sml{local}.

\begin{lstlisting}[style = SML, caption = {Definizione della funzione \sml{eval}}]
val rec cerca = fn s => fn [] => 0
						 | (s1, v)::l => if s1 = s then v else cerca s l;

val rec eval = fn env =>
			   fn costante    n           => n
			(*@ \phantom{t} @*) +++| variabile   s           => cerca s env+++
				| somma       (a1, a2)    => (eval env a1) +   (eval env a2)
				| sottrazione (a1, a2)    => (eval env a1) -   (eval env a2)
				| prodotto    (a1, a2)    => (eval env a1) *   (eval env a2)
				| divisione   (a1, a2)    => (eval env a1) div (eval env a2)
			(*@ \phantom{t} @*) +++| var (v, e1, e2) => eval ((v, eval env e1)::env) e2;+++
\end{lstlisting}

\subsection{commento della soluzione}

Gli operatori \sml{+}, \sml{-}, \sml{*}. \sml{div} si preoccupano di implementare la logica matematica del programma.
