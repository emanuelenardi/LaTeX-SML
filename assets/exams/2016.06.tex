\section{Giugno 2016}

\subsection{Turno 1}

\subsubsection{Testo}

Si scriva una funzione \sml{hist} (avente tipo \sml{real list -> real * real -> int}) che riceve come argomento \emph{una lista} di \sml{real l} ed \emph{una coppia} di \sml{real (c, d)}.
La funzione \sml{hist} ritorna \emph{il numero di elementi della lista compresi nell'intervallo \sml{(c - d, c + d)}}, estremi esclusi (vale a dire il numero di elementi \sml{r} tali che \sml{(c - d < r < c + d)}.

\medskip
Come esempio, l'invocazione

\begin{lstlisting}
hist [0.1, 0.5, 1.0, 3.0, 2.5] (1.0, 0.5);
\end{lstlisting}

deve avere risultato 1;

\begin{lstlisting}
hist [0.1, 0.5, 1.0, 3.0, 2.5] (1.0, 0.6);
\end{lstlisting}

deve avere risultato 2.

\subsubsection{Soluzione}

\begin{lstlisting}[style = SML, caption = {Definizione della funzione \sml{hist}}]
val rec hist = fn +++[]+++	   => (‡fn (c:real, d:real)‡ => 0)
				| +++[e]+++	   => (‡fn (c:real, d:real)‡ =>
											if (e > (c-d) andalso e < (c+d)) then
							 				(*@ \phantom{t} @*) Š1Š
											else
											(*@ \phantom{t} @*) Š0Š)
				| +++(e :: l)+++ => (‡fn (c:real, d:real)‡ =>
											if (e > (c-d) andalso e < (c+d)) then
											(*@ \phantom{t} @*) Š1 + hist l (c, d)Š
											else
											(*@ \phantom{t} @*) Š0 + hist l (c, d)Š);

val hist = fn: real list -> real * real -> int
\end{lstlisting}

\subsection{Commento della soluzione}

Vedi luglio e agosto '15.

% ---------------------------------------------------------------------------- %

\subsection{Turno 2}

\subsubsection{Testo}

Si scriva una funzione \sml{noduplen} (avente tipo \sml{''a list -> int}) che riceve come argomento una lista di \sml{''a l}.
La funzione \sml{noduplen} ritorna il numero di elementi della lista senza considerare i duplicati.

\medskip
Come esempio, l'invocazione

\begin{lstlisting}
noduplen ["pera", "pera", "pera", "pera"];
\end{lstlisting}

deve avere risultato 1;

\begin{lstlisting}
noduplen ["red", "red", "green", "blue"];
\end{lstlisting}

deve avere risultato 3.

\subsubsection{Soluzione}

\begin{lstlisting}[style = SML, caption = {Definizione della funzione \sml{noduplen}}]
val rec noduplen = fn []		=> 0
					| [a]		=> 1
					| a::(b::l)	=> if (a <> b) then
										1 + noduplen (b::l)
									else
										0 + noduplen (b::l);

val noduplen = fn: ''a list -> int
\end{lstlisting}

\subsection{Commento della soluzione}

Vedi luglio e agosto '15.
Da notare che la soluzione funziona soltanto su liste ordinate. Per una soluzione che funziona a prescindere dall'ordinamento, vedi giugno '18.
