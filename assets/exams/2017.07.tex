\section{Luglio 2017}

\subsection{Turno 1}

\subsubsection{Testo}

Si consideri il tipo di dato \texttt{FOR = For of int * (int -> int);} i cui valori \texttt{For(n, f)} rappresentano funzioni che implementano un ciclo for come il seguente:

\begin{cppcode}
int ciclofor (int x) {
	for (int i = 0; i < n; i++) {
		x = f(x);
	}
}
\end{cppcode}

Si scriva una funzione \texttt{eval} (avente tipo \texttt{FOR -> (int -> int)}) che riceve come argomento un valore di tipo \texttt{FOR} e ritorna una funzione da interi ad interi che implementa il ciclo indicato qui sopra (applica \texttt{n} volte la funzione \texttt{f} all'argomento).

\medskip
Come esempio, se \texttt{val f = fn x => x * 2}, allora \texttt{eval (For(3, f))} ritornerà una funzione che dato un numero \texttt{i} ritorna \texttt{i * 8}:

\begin{smlcode}
> val f = fn x => x * 2;
val f = fn: int -> int

> eval (For(3, f));
val it = fn: int -> int

> val g = eval (For(3, f));
val g = fn: int -> int

> g 5;
val it = 40: int
\end{smlcode}

\subsubsection{Soluzione}

\begin{listing}{!h}
\smlfile{assets/codes/2017.07/eval-0.sml}
\caption[]{Definizione della funzione \texttt{eval}}
\end{listing}

\subsection{Commento della soluzione}

TODO

\subsection*{Esempio di esecuzione}

\subsection{Turno 2}

\subsubsection{Testo}

Si consideri il tipo di dato \texttt{FOR = For of int * (int -> int);} i cui valori \texttt{For(n, f)} rappresentano funzioni che implementano un ciclo \texttt{for} come il seguente:

\begin{cppcode}
int ciclofor (int x) {
	for (int i = 0; i < n; i++) {
		x = f(x);
	}
}
\end{cppcode}

Si scriva una funzione \texttt{eval} (avente tipo \texttt{FOR -> (int -> int)}) che riceve come argomento un valore di tipo \texttt{FOR} e ritorna una funzione da interi ad interi che implementa il ciclo indicato qui sopra (applica \texttt{n - 1} volte la funzione \texttt{f} all'argomento).

\medskip
Come esempio, se \texttt{val f = fn x => x * 2}, allora \texttt{eval (For(3, f))} ritornerà una funzione che dato un numero \texttt{i} ritorna \texttt{i * 4}:

\begin{smlcode}
> val f = fn x => x * 2;
val f = fn: int -> int

> eval (For(3, f));
val it = fn: int -> int

> val g = eval (For(3, f));
val g = fn: int -> int

> g 5;
val it = 20: int
\end{smlcode}

\subsubsection{Soluzione}

\begin{listing}{!h}
\smlfile{assets/codes/2017.07/eval-1.sml}
\caption[]{Definizione della funzione \texttt{eval}}
\end{listing}

\subsection{Commento della soluzione}

TODO

\subsection*{Esempio di esecuzione}

TODO
