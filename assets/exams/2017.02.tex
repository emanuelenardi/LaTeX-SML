\section{Febbraio 2017}

\subsection{Testo d'esame}

Si implementi la funzione \texttt{unione}, avente tipo \texttt{insiemediinteri -> insiemediinteri -> insiemediinteri}, che dati due insiemi di interi ne calcola l'unione.

\subsection{Soluzione}

\begin{listing}{!h}
\smlfile{assets/codes/2017.02/unione.sml}
\caption[]{Definizione della funzione \texttt{unione}}
\end{listing}

\subsection{Commento della soluzione}

L'esercizio di risolve in modo analogo a quello dell'appello dell'agosto del 2016.

\subsection*{Esempio di esecuzione}

TODO
