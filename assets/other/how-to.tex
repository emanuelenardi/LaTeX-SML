\section*{Come leggere questa dispensa}

\subsection*{Trial and Error}

Il Trial and Error è un modo comune e veramente efficace per imparare. Al posto di chiedere aiuto su ogni piccola cosa, qualche volta spendere un po' di tempo da soli (a volte ore e giorni) e provare a far andare qualcosa ti aiuterà ad imparare più velocemente.

Se provi qualcosa e ti dà un errore, studia quell'errore. Quindi prova a correggere il tuo codice. Quindi prova a eseguirlo di nuovo. Se ricevi ancora un errore, modifica ancora il tuo codice. Continua a provare e fallire finché il tuo codice non fallisce più. Imparerai molto in questo modo leggendo questa dispensa, leggendo gli errori e imparando cosa funziona e cosa no. Provare, fallire, provare, fallire, provare, provare, provare, fallire, fallire, avere successo!

Questo è quanto hanno imparato molti "pros". Ma non aver paura di chiedere aiuto, noi non mordiamo (duro). L'apprendimento richiede tempo, i professionisti che hai incontrato non hanno imparato a diventare maestri in poche ore o giorni.

\subsection*{Indentazione}

L'indentazione è veramente importante! Il tuo codice funzionerà perfettamente senza, ma proocherà un groso mal di testa a te e agli altri leggere il tuo codice.

Il codice piccolo (25 linee o meno) probabilmente andrà bene senza indentazione, ma presto diventerà sciatto. È bene imparare ad indentare correttamente ASAP. L'indentazione non ha uno stile definito, ma è meglio mantenere tutto coerente.

\subsection*{Chiedere aiuto}

Before you ask, try doing some research yourself or try to code it yourself. If that did not yield results that satisfy you, read below.

\begin{itemize}
	\item Non essere preoccupato di chiedere aiuto, anche le persone più intelligenti chiedono aiuto agli altri;
	\item Non essere preoccupato di mostrare quello che hai provato, anche se pensi che sia stupido (in particolare in questo caso, potresti aver trovato un modo più semplice di risolvere il problema);
	\item Posta qualsiasi cosa tu abbia provato;
	\item Pretend everyone but you is a doorknob and knows nothing. Give as much information as you can to educate us doorknobs at what you are trying to do;
	\item Aiutaci aiutati;
	\item Sii paziente;
	\item Sii educato;
	\item Sii apero;
	\item Sii gentile;
	\item Buon divertimento!
\end{itemize}
