\section{Giugno 2017}

\subsection{1\textsuperscript{o} Turno}

\subsubsection{Testo}

Si scriva una funzione \sml{sommali} (avente tipo \sml{int -> int list -> int}) che riceve come argomento un intero \sml{n} ed una lista di interi \sml{l}.
La funzione \sml{sommali} somma ad \sml{n} gli elementi di \sml{l} che hanno posizione \emph{pari} (se la lista contiene meno di 2 elementi, \sml{sommali} ritorna \sml{n}).

\medskip
Come esempio, l'invocazione

\begin{lstlisting}
sommali 0 [1,2];
\end{lstlisting}

deve avere risultato 2;

\begin{lstlisting}
sommali 1 [1,2,3];
\end{lstlisting}

deve avere risultato 3;

\begin{lstlisting}
sommali 2 [1,2,3,4];
\end{lstlisting}

deve avere risultato 8.

\subsubsection{Soluzione}

\begin{lstlisting}[style = SML, caption = {[Definizione della funzione \sml{sommali} - 1\textsuperscript{o} Turno]Definizione della funzione \sml{sommali}}]
val rec sommali = fn z => fn []     => z
						| v::[]     => z
						| v1::v2::l => v2 + (sommali z l);

val sommali = fn: int -> int list -> int
\end{lstlisting}

\subsection{2\textsuperscript{o} Turno}

\subsubsection{Testo}

Si scriva una funzione \sml{sommali} (avente tipo \sml{int -> int list -> int}) che riceve come argomento un intero \sml{n} ed una lista di interi \sml{l}.
La funzione \sml{sommali} somma ad \sml{n} gli elementi di \sml{l} che hanno posizione \emph{multipla di 3} (se la lista contiene meno di 3 elementi, \sml{sommali} ritorna \sml{n}).

\medskip
Come esempio, l'invocazione

\begin{lstlisting}
sommali 0 [1,2,3];
\end{lstlisting}

deve avere risultato 3,

\begin{lstlisting}
sommali 1 [1,2,3];
\end{lstlisting}

deve avere risultato 4: e

\begin{lstlisting}
sommali 2 [1,2,3,4,5,6];
\end{lstlisting}

deve avere risultato 11.

\subsubsection{Soluzione}

\begin{lstlisting}[style = SML, caption = {[Definizione della funzione \sml{sommali} - 2\textsuperscript{o} Turno]Definizione della funzione \sml{sommali}}]
val rec sommali = fn z => fn []          => z
						| v::[]          => z
						| v1::v2::[]     => z
						| v1::v2::v3::l  => v3 + (sommali z l);

val sommali = fn: int -> int list -> int
\end{lstlisting}
