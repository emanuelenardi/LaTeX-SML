\begin{ocg} [printocg = never] {Notprinted} {noprint} {1}
\section*{Introduzione}

Lo scopo principale di questi appunti è quello di esaminare più da vicino gli esami di programmazione funzionale tenuti all'Università degli Studi di Trento. Queste note non sono complete, e la loro lettura non permette, da sola, di superare l’esame. La versione più recente di queste note si trova all'indirizzo:

\begin{center}
	\url{https://github.com/emanuelenardi/LaTeX-SML}
\end{center}

Il materiale didattico trattato in questo documento si trova nella cartella Google Drive del corso di Informatica %
\href{https://bit.ly/drive-folder}{\ExternalLink}.

\medskip
Hai trovato un errore? Inviami un'e-mail \href{mailto:emanuele.nardi@studenti.unitn.it}{\ExternalLink} o contattami direttamante su Telegram \href{https://t.me/emanuelenardi}{\ExternalLink}.

\medskip
Ulteriori contatti dell'ateneo si trovano sulla pagina del DISI %
\href{http://offertaformativa.unitn.it/it/l/informatica/contatti-e-referenti}{\ExternalLink}.

\tableofcontents					% produce l’indice generale
\lstlistoflistings					% produce l'elenco dei codici
% \listoffigures					% produce l'elenco delle figure
% \listoftables						% produce l'elenco delle tabelle
%
% \newpage
% \listoftodos[Note]					% produce l'elenco delle cose da fare
% \cofeAm{1}{0.75}{0}{5.5cm}{3cm}		% produce un macchia di caffé
\end{ocg}
