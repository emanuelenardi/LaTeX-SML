\section{Giugno 2017}

\subsection{Turno 1}

\subsubsection{Testo}

Si scriva una funzione \texttt{sommali} (avente tipo \texttt{int -> int list -> int}) che riceve come argomento un intero \texttt{n} ed una lista di interi \texttt{l}.
La funzione \texttt{sommali} somma ad \texttt{n} gli elementi di \texttt{l} che hanno posizione \emph{pari} (se la lista contiene meno di 2 elementi, \texttt{sommali} ritorna \texttt{n}).

\medskip
Come esempio, l'invocazione

\begin{smlcode}
sommali 0 [1,2];
\end{smlcode}

deve avere risultato 2;

\begin{smlcode}
sommali 1 [1,2,3];
\end{smlcode}

deve avere risultato 3;

\begin{smlcode}
sommali 2 [1,2,3,4];
\end{smlcode}

deve avere risultato 8.

\subsubsection{Soluzione}


\begin{listing}{!h}
\smlfile{assets/codes/2017.06/sommali-1.sml}
\caption[funzione \texttt{sommali} -- Turno 1]{Definizione della funzione \texttt{sommali}}
\end{listing}

\subsection{Commento della soluzione}

Vedi agosto '15, giugno '16.

\subsection*{Esempio di esecuzione}

\subsection{Turno 2}

\subsubsection{Testo}

Si scriva una funzione \texttt{sommali} (avente tipo \texttt{int -> int list -> int}) che riceve come argomento un intero \texttt{n} ed una lista di interi \texttt{l}.
La funzione \texttt{sommali} somma ad \texttt{n} gli elementi di \texttt{l} che hanno posizione \emph{multipla di 3} (se la lista contiene meno di 3 elementi, \texttt{sommali} ritorna \texttt{n}).

\medskip
Come esempio, l'invocazione

\begin{smlcode}
sommali 0 [1,2,3];
\end{smlcode}

deve avere risultato 3,

\begin{smlcode}
sommali 1 [1,2,3];
\end{smlcode}

deve avere risultato 4: e

\begin{smlcode}
sommali 2 [1,2,3,4,5,6];
\end{smlcode}

deve avere risultato 11.

\subsubsection{Soluzione}

\begin{listing}{!h}
\smlfile{assets/codes/2017.06/sommali-2.sml}
\caption[funzione sommali -- turno 2]{Definizione della funzione \texttt{sommali}}
\end{listing}

\subsection{Commento della soluzione}

Vedi agosto '15, giugno '16.

\subsection*{Esempio di esecuzione}
