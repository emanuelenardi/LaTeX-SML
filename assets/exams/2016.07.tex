\section{Luglio 2016}

\subsection{Turno 1}

\subsubsection{Testo}

Si consideri il tipo di dato
\begin{listing}{!h}
\smlfile{assets/codes/2016.07/lambda.sml}
\caption[]{Definizione del tipo di dato \texttt{espressione Lambda}}
\end{listing}
che rappresenta un'espressione del Lambda-calcolo.

\medskip
Il costruttore \texttt{Var} crea un'espressione costituita da un'unica funzione/variabile (il cui nome è un valore di tipo \texttt{string});
il costruttore \texttt{Lambda} crea una Lambda-espressione a partire da un'altra espressione, legandone una variabile (indicata da un valore di tipo string);
il costruttore \texttt{Apply} crea un'espressione data dall'applicazione di un'espressione ad un'altra.

\medskip
Si scriva una funzione \texttt{is_free} (avente tipo \texttt{string -> lambda_expr -> bool}) che riceve come argomenti una stringa (che rappresenta il nome di una variabile / funzione) ed una Lambda-espressione, %
ritornando \texttt{true} se \emph{la variabile indicata appare come libera nell'espressione}, %
\texttt{false} altrimenti (quindi, la funzione ritorna false se la variabile è legata o se non appare nell'epressione).

\medskip
Come esempio, l'invocazione

\begin{smlcode}
is_free "a" (Var "a")
\end{smlcode}

deve avere risultato true, l'invocazione

\begin{smlcode}
is_free "b" (Var "a")
\end{smlcode}

deve avere risultato false, l'invocazione

\begin{smlcode}
is_free "a" (Lambda ("a", Apply((Var "a"), Var "b")))
\end{smlcode}

deve avere risultato false, l'invocazione

\begin{smlcode}
is_free "b" (Lambda ("a", Apply((Var "a"), Var "b")))
\end{smlcode}

deve avere risultato true e così via.

\medskip
\textbf{IMPORTANTE}: notare il tipo della funzione! La funzione usa la \emph{tecnica del currying} per gestire i due argomenti.

\subsubsection{Soluzione}

\begin{listing}{!h}
\begin{smlcode}
datatype lambda_expr = Var of string
					 | Lambda of string * lambda_expr
					 | Apply of lambda_expr * lambda_expr;
\end{smlcode}
\caption[]{Definizione del tipo di dato \texttt{espressione Lambda}}
\end{listing}

\begin{listing}{!h}
\begin{smlcode}
val rec is_free = fn s => fn Var v => if (s = v) then
									      true
									  else
									      false
				| Lambda (v, e) => if (s = v) then
									   false
								   else
									   is_free s e
				| Apply (e1, e2) => (is_free s e1) orelse (is_free s e2);

val is_free = fn: string -> lambda_expr -> bool
\end{smlcode}
\caption{Definizione della funzione \texttt{is_free}}
\end{listing}

\subsection{Commento della soluzione}

TODO

\subsection{Turno 2}

\subsubsection{Testo}

Basandosi sul tipo di dato \texttt{espressione} e la funzione \texttt{eval} definiti come segue:

\begin{listing}{!h}
\smlfile{assets/codes/2016.07/semplifica.sml}
\caption{Definizione della funzione \texttt{semplifica}}
\end{listing}

il tipo espressione può essere esteso come segue per supporare il concetto di variabile:

\begin{listing}{!h}
\smlfile{assets/codes/2016.07/espressione.sml}
\caption{Definizione del tipo di dato \texttt{espressione}}
\end{listing}

Si riscriva la funzione \texttt{eval} per supportare i due nuovi costruttori \texttt{variabile} e \texttt{var. variabile x}, con \texttt{x} di tipo stringa, é valutata al valore della variabile di nome \texttt{x} (per fare questo, \texttt{eval} deve cercare nell'ambiente un legame fra tale nome ed un valore). %
\texttt{var (x, e1, e2)} é valutata al valore di \texttt{e2} dopo aver assegnato ad \texttt{x} il valore di \texttt{e1}.

\medskip
Per poter valutare correttamente \texttt{variabile} e \texttt{var}, \texttt{eval} deve quindi ricevere come argomento l'ambiente in cui valutare le variabili. %
Tale ambiente può essere rappresentato come una lista di coppie (stringa, intero) ed avra' quindi tipo \texttt{(string * int) list}.

\medskip
La funzione \texttt{eval} deve quindi avere tipo \texttt{(string * int) list -> espressione -> int}.

\subsubsection{Soluzione}

Questa é una possibile soluzione. %
Si noti che in questa soluzione la funzione \texttt{cerca} viene definita come visibile a tutti, mentre sarebbe più opportuno renderla locale a \texttt{eval} usando un costrutto \texttt{let} o \texttt{local}.

\begin{listing}{!h}
% [highlightlines={6,11}]
\begin{smlcode}
val rec cerca = fn s => fn [] => 0
						 | (s1, v)::l => if s1 = s then v else cerca s l;

val rec eval = fn env =>
			   fn costante    n           => n
				| variabile   s           => cerca s env
				| somma       (a1, a2)    => (eval env a1) +   (eval env a2)
				| sottrazione (a1, a2)    => (eval env a1) -   (eval env a2)
				| prodotto    (a1, a2)    => (eval env a1) *   (eval env a2)
				| divisione   (a1, a2)    => (eval env a1) div (eval env a2)
				| var (v, e1, e2) => eval ((v, eval env e1)::env) e2;
\end{smlcode}
\caption[]{Definizione della funzione \texttt{eval}}
\end{listing}

\subsection{Commento della soluzione}

Gli operatori \texttt{+}, \texttt{-}, \texttt{*}. \texttt{div} si preoccupano di implementare la logica matematica del programma.

\subsection*{Esempio di esecuzione}

TODO
