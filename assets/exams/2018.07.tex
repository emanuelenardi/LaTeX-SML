\section{Luglio 2018}
\subsection{Testo}

Si consideri il tipo di dato \sml{ITER = Iter of int * (int -> int);} i cui valori \sml{Iter(n, f)} rappresentano funzioni che implementano un ciclo for come il seguente:

\begin{lstlisting}
int ciclofor (int x) {
	for (int i = 0; i < n; i++) {
		x = f(x);
	}
}
\end{lstlisting}

Si scriva una funzione \sml{eval} (avente tipo \sml{ITER -> (int -> int)}) che riceve come argomento un valore di tipo \sml{ITER} e ritorna una funzione da interi ad interi che implementa il ciclo indicato qui sopra (applica \sml{n} volte la funzione \sml{f} all'argomento).

\medskip
Come esempio, se \sml{val f = fn x => x + 2}, allora \sml{eval (Iter(8, f))} ritornerà una funzione che dato un numero \sml{i = 0} ritorna \sml{i = 16}:

\begin{lstlisting}[style = SML]
> val f = fn x => x + 2;
val f = fn: int -> int

> eval (Iter(8, f));
val it = fn: int -> int

> val g = eval (Iter(8, f));
val g = fn: int -> int

> g 5;
val it = 21: int
\end{lstlisting}

\subsubsection{Soluzione}

\begin{lstlisting}[style = SML, caption = {definizione della funzione \sml{eval}}]
datatype ITER = Iter of int * (int -> int);

val rec eval = fn Iter (n, f) =>
		fn x => if (n > 0) then
					eval (Iter (n - 1, f)) (f x)
				else
					x;
\end{lstlisting}

\subsection{Commento della soluzione}
Vedi luglio '17.