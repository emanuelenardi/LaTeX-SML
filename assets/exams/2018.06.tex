\section{Giugno 2018}

\subsection{Testo d'esame}

Si scriva una funzione \texttt{conta} (avente tipo \texttt{''a list -> int}) che riceve come argomento una lista di \texttt{''a l}.
La funzione \texttt{conta} ritorna il numero di elementi della lista senza considerare i duplicati.

\medskip
Come esempio, l'invocazione

\begin{smlcode}
conta ["uno", "uno", "uno", "uno"];
\end{smlcode}

deve avere risultato 1;

\begin{smlcode}
conta [1, 2, 2, 3];
\end{smlcode}

deve avere risultato 3;

\begin{smlcode}
conta [2, 1, 3, 2];
\end{smlcode}

deve avere risultato 3.

\subsubsection{Soluzione}

\begin{listing}{!h}
\smlfile{assets/codes/2018.06/conta.sml}
\caption{Definizione della funzione \texttt{conta}}
\end{listing}

\subsection{Commento della soluzione}

Da notare che negli esami passati non era stata usata nessuna funzione di libraria (come \texttt{List.exists}) in quanto il Professor Abeni non permetteva di utilizzarle.

\subsection*{Esempio di esecuzione}

TODO
