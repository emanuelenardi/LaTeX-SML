\section{Settembre 2015}

\subsection{Testo}

Si consideri il seguente tipo di dato:

\begin{lstlisting}[style = SML, caption = {Definizione del tipo di dato \sml{codice}}]
datatype codice = rosso of string
				| giallo of string
				| verde of string;
\end{lstlisting}

che rappresenta un paziente in arrivo al pronto soccorso.

\medskip
La stringa rappresenta il cognome del paziente, mentre i tre diversi costruttori \sml{rosso}, \sml{giallo} e \sml{verde} rappresentano la gravità del paziente (codice \sml{rosso}: massima gravità/urgenza, codice \sml{verde}: minima gravità/urgenza).

\medskip
Quando un paziente con codice \sml{rosso} arriva al pronto soccorso, viene messo in lista d'attesa dopo tutti i pazienti con codice \sml{rosso} (ma prima di quelli con codice \sml{giallo} o verde); quando arriva un paziente con codice \sml{giallo}, viene messo in lista d'attesa dopo tutti i pazienti con codice \sml{rosso} o \sml{giallo} (ma prima di quelli con codice \sml{verde}), mentre quando arriva un paziente con codice \sml{verde} viene messo in lista d'attesa dopo tutti gli altri pazienti.

\medskip
Si scriva una funzione arriva (avente tipo codice \sml{list -> codice -> codice list}) che riceve come argomenti la lista dei pazienti in attesa (lista di elementi di tipo codice) ed un paziente appena arrivato (elemento di tipo codice) e ritorna la lista aggiornata dei pazienti in attesa (dopo aver inserito il nuovo paziente nel giusto posto in coda).

\medskip
Come esempio, l'invocazione

\begin{lstlisting}
arriva [rosso "topolino", rosso "cip", giallo "ciop", verde "paperino", verde "pluto"] (giallo "clarabella");}
\end{lstlisting}

deve avere risultato

\begin{lstlisting}
[rosso "topolino", rosso "cip", giallo "ciop", giallo "clarabella", verde "paperino", verde "pluto"]}
\end{lstlisting}

\medskip
\textbf{IMPORTANTE}: notare il tipo della funzione! Si noti inoltre che la funzione usa la \emph{tecnica del currying} per gestire i due argomenti.

\subsection{Soluzione}

\begin{lstlisting}[style = SML, nolol = true, caption = {Definizione del tipo di dato \sml{codice}}]
datatype codice = rosso of string
				| giallo of string
				| verde of string;
\end{lstlisting}

\begin{lstlisting}[style = SML, caption = {Definizione della funzione \sml{arriva}}]
val rec arriva = fn
		    []			  => (fn x => [x])
		  | (verde  n)::l => (fn (verde  nn) => (verde  n)::(arriva l (verde nn))
							   | x           => x::((verde n)::l))
		  | (giallo n)::l => (fn (verde  nn) => (giallo n)::(arriva l (verde nn))
		  					   | (giallo nn) => (giallo n)::(arriva l (giallo nn))
							   | x           => x::((giallo n)::l))
		  | (rosso  n)::l => (fn x => (rosso n)::(arriva l x));

val arriva = fn: codice list -> codice -> codice list
\end{lstlisting}
